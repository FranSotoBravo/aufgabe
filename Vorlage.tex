% Dies ist eine Vorlage für Eure Bachelorarbeit an der RUB,
% erstellt von Alexander Noack, zur freien Verwendung ;-)

\RequirePackage[l2tabu,orthodox]{nag} % checke common mistakes/outdated pkgs
% Die KOMA Klasse für article mit einer 11pt Schrift auf A4 Papier
\documentclass[11pt,
               a4paper,
               parskip=half,
%               bibliography=totoc,
               ]{scrartcl}

% ============ %
% Pakete laden %
% ============ %

% Damit Trennregeln aktualisiert werden
\usepackage[ngerman=ngerman-x-latest]{hyphsubst} % als erstes laden!

% Input encoding auf utf8 setzen, und die Ausgabe in T1 kodieren (Westeuropa)
\usepackage[utf8]{inputenc} % in einer aktuellen LaTeX Version nicht nötig
\usepackage[T1]{fontenc}

% geometry -- praktisch, wenn man genaue Vorstellungen/Vorgaben des Layouts hat
\usepackage[pass]{geometry} % Option pass übergibt die Werte von KOMA

% Kopf- und Fußzeile verändern
\usepackage{scrlayer-scrpage}

% Nutze latin modern font und fixe ein paar Probleme
\usepackage{lmodern}
\usepackage{fixcmex}

% Mikro-Typographie
\usepackage{microtype}

% Deutsche Sprache mit _n_euer Rechtschreibung zusätzlich zu englisch
\usepackage[english,ngerman]{babel}

% Komfortables Titel-, Autor-, .. -Handling
\usepackage{titling}

% Um zum Beispiel Grafiken einzubinden
\usepackage[usenames,dvipsnames]{xcolor}
\usepackage{graphicx} % lädt auch xcolor 

% Mathematik Pakete und extra Symbole
%\usepackage{amsmath}
\usepackage{mathtools} % lädt auch amsmath
\usepackage{amssymb}
\usepackage{textcomp}
\usepackage{gensymb}

% Ermöglicht komfortabel das Anpassen von Abständen, Labels etc. für itemize
\usepackage{enumitem}

% Für mehrere Bilder die zusammengehören praktisch
\usepackage{subcaption}

% Lade Tabellen Pakete
\usepackage{array}
\usepackage{booktabs}
\usepackage{multirow}
%\usepackage{tabularx}
%\usepackage{longtable}
\usepackage{csvsimple}

% Um Quellcode einzubinden
\usepackage{listings}

% Um mit LaTeX zu zeichnen und zu plotten
%\usepackage{tikz}
%\usepackage{pgfplots} % lädt auch tikz

% Physik- und formel- oder symbolbezogene Pakete
\usepackage{siunitx}
\usepackage{physics}
%\usepackage{braket}
\usepackage[thinc]{esdiff}

% Erleichtern das Leben beim Editieren, Probelesen und Arbeiten
\usepackage{blindtext}
\usepackage[colorinlistoftodos]{todonotes}
\usepackage{lineno}

% Paket für wörtliche Zitate, URLs, einstellbaren Zeilenabstand
\usepackage{csquotes}
\usepackage{url}
\usepackage{setspace}

% Pakete, um User-defined-Macros zu erstellen, oder vorhandene zu verändern
\usepackage{xspace}
%\usepackage{xparse}

% für anklickbare Links (cross-referencing) ins Dokument und nach außen.
\usepackage[hidelinks,
            linktocpage=false,
            pdfusetitle]{hyperref} % als letztes (spät, exceptions) laden!
\usepackage{bookmark} % als letzteres laden, Optimierungen zu hyperref, \pdfbookmark

% =============== %
% Ende für Pakete %
% =============== %


% User-defined macros
\newcommand{\zB}{z.\,B.\xspace}
\newcommand{\ZB}{Z.\,B.\xspace}
\newcommand{\textsw}[1]{\texttt{#1}} % sw=software
\newcommand{\file}[1]{\texttt{#1}} % file names

% Header, Footer
\ihead{\thetitle} % inner head (einseitig links)
\chead{} % center head
\ohead{\rightmark} % outer head (einseitig rechts)
\automark{section} % setze \rightmark auf section name
\automark*{subsection} % falls subsection vorhanden auf diese
\setkomafont{pagehead}{\sffamily} % Kopfzeilen-Font

% list related (itemize, enumerate)
\setlist{nosep} % entferne alle Abstände
\setitemize[1]{label=\raisebox{.37ex}{\scalebox{.6}{$\bullet$}}} % kleinere bullet

% math related
\AtBeginDocument{% verkleinere Raum um Mathe Umgebungen
  \setlength{\abovedisplayskip}{6pt plus 3pt minus 3pt}%=11pt plus 3pt minus 6pt
  \setlength{\abovedisplayshortskip}{0pt plus 3pt}%=0pt plus 3pt
  \setlength{\belowdisplayskip}{6pt plus 3pt minus 3pt}%=11pt plus 3pt minus 6pt
  \setlength{\belowdisplayshortskip}{4pt plus 3pt minus 3pt}%=6.5pt plus 3.5pt minus 3pt
}
\newcolumntype{L}{>{$}l<{$}} % math-mode version of "l" column type
\newcolumntype{C}{>{$}c<{$}} % math-mode version of "c" column type

% table related
\setlength{\cmidrulekern}{.4em}

% Neue Einheiten für siunitx
\DeclareSIUnit{\erg}{erg}
\DeclareSIUnit{\Gauss}{G}

% Titel, Autor, etc
\title{Beispieldokument EWA} % Titel in Sprache der Arbeit
\newcommand{\theothertitle}{Example Document} % Titel in anderer Sprache
\newcommand{\bachelormaster}{Bachelor} % {Bachelor} oder {Master}
\newcommand{\sciencearts}{Science} % {Science} oder {Arts}
\author{Vorname Name} % Autor
\newcommand{\placeofbirth}{Deutschland} % Geburtsort
\newcommand{\location}{Bochum}
\date{2018} % Datum (Jahr)
\usepackage{lipsum}

  %zu Aufgabe 8, 9
 \usepackage[backend=biber, natbib=true, sorting=nyt, url=false,
 doi=false, isbn=false, maxbibnames=3, maxcitenames=2, date=year,
 style=authoryear, citestyle=authoryear-comp]{biblatex}
 \bibliography{literatur.bib}

 \newcommand {\nd }[2]{\frac {\mathrm {d}^#1 }{\mathrm {d} #2^#1}}
 \newcommand {\ToDo }[1]{{\color{red}To do: #1}}
%8.c
\usepackage[disable]{\ToDo}



% ================= %
% Ende der Preamble %
%  Beginn Dokument  %
% ================= %

\begin{document}


% Titelseite
% Benötigte Pakete in dieser Version:
% geometry, hyperref, bookmark, titling, setspace
\newgeometry{margin=2.5cm,bottom=4cm}%,bindingoffset=6mm}
\pdfbookmark[section]{Titelseite}{titlepage}
\begin{titlepage}
  \centering
  {\huge\titlefont\thetitle\par
                  \bigskip\bigskip
                  \theothertitle\par}
  \vspace{2cm}

  \begin{spacing}{0.8}
    {\LARGE \bachelormaster arbeit\par
            \bigskip\medskip
            im Studiengang\par
            ,,\bachelormaster{} of \sciencearts``\par
            im Fach Physik\par
            \bigskip\medskip
            an der Fakultät für Physik und Astronomie\par
            der Ruhr-Universität Bochum\par}

    \vfill

    {\LARGE von\par
            \theauthor\par
            \bigskip\medskip
            aus\par
            \placeofbirth\par}
  \end{spacing}

  \vspace{1.8cm}

  {\LARGE \location{} \thedate\par}
\end{titlepage}
\restoregeometry
\cleardoublepage

% Table of Contents
\pdfbookmark[section]{\contentsname}{toc}
\tableofcontents
\cleardoublepage


\section{Übungsblatt 4}

\subsection{Aufgabe 1}

I actually had a very big problem with this section: i wasn't able to download the software and try the program myself (I don't think we really have a "Südpol" in my host university). I did ask a friend to show me what he did in Südpol, and I Think it went something like this:\\
1. Look for the paper and the graph \\
2. Define the axis in the program\\
3. Create a curve\\
4. save\\

\subsection{Aufgabe 3]

For this part I will copy-paste the commands from my terminal (I have a Mac, and used an old version of Xcode):\\


\subsubsection{Part a}

FrancisavosMBP3:~ fran$ git config --global user.name "Francisca Soto" \\
FrancisavosMBP3:~ fran$ git config --global user.email "francisca.sotobravo@rub.de\\
\midskip

FrancisavosMBP3:~ fran$ git init Bachelor_test\\
Initialized empty Git repository in /Users/fran/Bachelor_test/.git/\\

\subsubsection{Part b}

FrancisavosMBP3:Bachelor_test fran$ open .gitignore \\

For my .gitignore I used a list of recomemended options from a Overleaf-help website; I had:\\
*.log\\
*.com\\
*.exe\\
*.aux\\
*.glo\\
*.idx\\
*.log\\
*.toc\\
*.ist\\
*.acn\\
*.acr\\
*.alg\\
*.bbl\\
*.blg\\
*.dvi\\
*.glg\\
*.gls\\
*.ilg\\
*.ind\\
*.lof\\
*.lot\\
*.maf\\
*.mtc\\
*.mtc1\\
*.out\\
*.synctex.gz\\

\subsubsection{Part c}

I copied and pasted the .tex Vorlage from the past homework (was that what we were supposed to do??) with:\\
FrancisavosMBP3:Bachelor_test fran$ git add Vorlage.tex\\
FrancisavosMBP3:Bachelor_test fran$ git commit -m "My first commit"\\
[master (root-commit) 13c2970] My first commit\\
 1 file changed, 929 insertions(+)\\
 create mode 100644 Vorlage.tex\\
 
 \subsubsection{Part d}
 
 FrancisavosMBP3:Bachelor_test fran$ git checkout -b First-Branch\\
Switched to a new branch 'First-Branch'

\subsubsection{Part e - inception}

I created an account at github.com, created a new repository and then pull my local one: \\
FrancisavosMBP3:Bachelor_test fran$ git remote add aufgabe https://github.com/FranSotoBravo/aufgabe.git\\
FrancisavosMBP3:Bachelor_test fran$ git push -u aufgabe master\\
Username for 'https://github.com': FranSotoBravo\\
Password for 'https://FranSotoBravo@github.com': \\
Counting objects: 3, done.\\
Delta compression using up to 4 threads.\\
Compressing objects: 100% (2/2), done.\\
Writing objects: 100% (3/3), 14.44 KiB | 0 bytes/s, done.\\
Total 3 (delta 0), reused 0 (delta 0)\\
To https://github.com/FranSotoBravo/aufgabe.git\\
 * [new branch]      master -> master\\
Branch master set up to track remote branch master from aufgabe.\\

\midskip

Then I used the edit option in github to erase the previous document (I mean, alle Teile waren irrelevant, oder??) and write this precise words you are currently reading (enjoy your weekend btw)
 
\subsubsection{f}

Here I used the values my friend provided for me:\\

Radio		Jet	
20	302	42	6253
20	6927	40	7273
20	7638	42	6253
20	9591	41	9867
21	1012	42	4462
21	119	42	4656
21	4565	42	222
21	8117	43	687
22	3801	42	9268
22	4156	42	9508
22	611	42	714
22	6821	41	9867
22	9307	43	3171
23	728	42	9978
23	4281	43	9734
23	7123	42	8381
23	8544	42	7494
24	142	45	7472
24	2274	44	8426
24	3694	45	1796
24	405	44	2217
24	4938	44	612
24	5826	45	8537
24	6359	44	7539
24	6536	44	2217
24	6892	45	1264
24	7247	44	7539
24	8135	45	909
24	849	44	167
24	8668	43	9557
25	799	44	1153
25	222	45	8537
25	2753	46	1747
25	3286	43	8315
27	6732	45	1086

\subsubsection{g}
git checkout + number \\

"git log" to see the history 
\end{document}
